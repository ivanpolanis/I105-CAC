\section{Segmentación de instrucciones}

En una arquitectura \textbf{RISC}\footnote{Reduced Instruction Set Computer} la mayoría de las instrucciones son del tipo registro a registro, y un ciclo de instrucción tiene las dos fases siguientes:

\begin{itemize}
  \item \textbf{I}: captación de instrucción
  \item \textbf{E}: ejecución. Realiza una operación de la ALU como entrada y salida.
\end{itemize}

Las operaciones de carga y almacenamiento necesitan tres fases:

\begin{itemize}
  \item \textbf{I}: captación de instrucción
  \item \textbf{E}: ejecución. Calcula una dirección de memoria.
  \item \textbf{D}: memoria. Operación registro a memoria o memoria a registro.
\end{itemize}

Dada la simplicidad y regularidad de un repertorio de instrucciones RISC, el diseño de la organización en tres o cuatro etapas se realiza fácilmente. 

\subsection{Segmentación en el WinMIPS64}

\begin{itemize}
  \item \textbf{Búsqueda (F, Fetch)}
  \begin{itemize}
    \item Se accede a memoria por la instrucción.
    \item Se incrementa el Program Counter.
  \end{itemize}
  \item \textbf{Decodificación (D, Decode)}
  \begin{itemize}
    \item Se decodifica la instrucción, obteniendo operación a realizar en la ruta de datos.
    \item Se accede al banco de registros por el/los operandos/s (si es necesario).
    \item Se calcula el valor del operando inmediato con extensión de signo (si es necesario).
  \end{itemize}
  \item \textbf{Ejecución (X, Execute)}
  \begin{itemize}
    \item Se ejecuta la operación en la ALU.\@
  \end{itemize}
  \item \textbf{Acceso a memoria (M, Memory Access)}
  \begin{itemize}
    \item Si se requiere un acceso a memoria, se accede.
  \end{itemize}
  \item \textbf{Almacenamiento (W, Writeback)}
  \begin{itemize}
    \item Si se requiere volcar un resultado a un registro, se accede al banco de registros.
  \end{itemize}
\end{itemize}

\subsection{Segmentación de cauce}

La segmentación de cauce (pipelining) es forma particularmente efectiva de organizar el hardware de la CPU para realizar más de una operación al mismo tiempo. 

Consiste en descomponer el proceso de ejecución de las instrucciones en fases o etapas que permitan una ejecución simultánea. Explota el paralelismo entre las instrucciones de un flujo secuencial.

La segmentación es una técnica de mejora de prestaciones a nivel de diseño hardware. La misma es invisible al programador.

Es necesario uniformizar las etapas al tiempo de la más lenta.

El diseño de procesadores segmentados tiene gran dependencia del repertorio de instrucciones.

La segmentación de cauce incrementa la productividad, pero no reduce el tiempo de ejecución de la instrucción.

\subsubsection{Atascos de un cauce}

Son situaciones que impiden a la siguiente instrucción que se ejecute en el ciclo que le corresponde. Estas pueden ser:

\begin{itemize}
  \item \textbf{Estructurales}
  \subitem{Provocados por conflictos por los recursos.}
  \item \textbf{Por dependencia de datos}
  \subitem{Ocurren cuando dos instrucciones se comunican por medio de un dato.}
  \item \textbf{Por dependencia de control}
  \subitem{Ocurren cuando la ejecución de una instrucción depende de cómo se ejecute otra.}
\end{itemize}

\subsection{Optimización de la segmentación}

Dada la naturaleza sencilla y regular de las instrucciones RISC, los esquemas de segmentación se pueden emplear eficientemente. Hay poca variación en la duración de la ejecución de instrucciones, y el cauce puede adaptarse para reflejar este hecho.

Para compensar las dependencias de datos, se han desarrollado técnicas de reorganización de código. Consideremos primero las instrucciones de salto. El \textit{salto retardado}, que es una forma de incrementar la eficiencia de la segmentación, utiliza un salto que no tiene lugar hasta después de que se ejecute la siguiente instrucción. La posición de la instrucción inmediatamente después de la instrucción de salto se conoce como \textit{espacio de retardo}.

Para saltos condicionales, el procedimiento no puede aplicarse a ciegas. Si la condición comprobada por la bifurcación puede alterarse por la instrucción inmediatamente precedente, el compilador ha de abstenerse de hacer el intercambio en su lugar, debe inserta un NOOP.\@

Un tipo de táctica similar, llamada \textit{carga retardada}, se puede usar con las instrucciones LOAD.\@ En las instrucciones LOAD, el procesador bloquea el registro destino de la carga. Después el procesador continúa la ejecución del flujo de instrucciones hasta que se alcanza una instrucción que necesite ese registro, deteniéndose hasta que la carga finalice. Si el compilador puede reorganizar las instrucciones de manera que se pueda hacer un trabajo útil mientras la carga está en el cauce, la eficiencia aumenta.\@

